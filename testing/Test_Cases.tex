\documentclass{article}

% --------------------
% Encoding & font
% --------------------
\usepackage[utf8]{inputenc}
\usepackage[T1]{fontenc}
\usepackage{lmodern}

% --------------------
% Page layout
% --------------------
\usepackage[a4paper, margin=2cm]{geometry}

% --------------------
% Title info
% --------------------
\title{Test Case Specification}
\author{Group 1}
\date{\today}

\begin{document}

\maketitle

\section{Introduction}
This document describes test cases for the LTU Search Engine project.
% ==========================================================
\section{Test Case: TC-FRQ-1001}

\subsection*{Related Requirements}
FRQ-1001

\subsection*{Description}
Given a set of linked URLs, Verify that the crawler starts from a seed URL then finds and visits all reachable web pages recursively. 

\subsection*{Preconditions}
\begin{itemize}
  \item Crawler component is operational
  \item A search index surrogate is created and will store visited URL pages and the order of the visits.
  \item The search index surrogate is initially empty.
  \item A source seed URL, with accompanied HTML page, and at least two additional HTML pages outside of, but linked by the seed URL. And finally in one of the two last pages link to a final page. 
\end{itemize}

\subsection*{Test Steps}
\begin{enumerate}
  \item Configure the crawler with the HTML page as the seed URL.
  \item Start the crawling process.
  \item Inspect the contents of the search index.
\end{enumerate}

\subsection*{Expected Results}
\begin{itemize}
  \item No errors occur.
  \item The search index surrogate contains entries from the seed URL page and all other linked pages in recursively order.
\end{itemize}

% ==========================================================

% ==========================================================
\section{Test Case: TC-FRQ-1007}

\subsection*{Related Requirements}
FRQ-1007

\subsection*{Description}
Verify that the crawler ignores non-relevant resources such as images, CSS files, and JavaScript files, 
and only processes HTML documents during crawling.

\subsection*{Preconditions}
\begin{itemize}
  \item The crawler component is operational.
  \item The crawler is configured with a seed URL pointing to an HTML page.
  \item The seed HTML page contains links to:
    \begin{itemize}
      \item One or more HTML pages
      \item Image resources (e.g., \texttt{.png}, \texttt{.jpg})
      \item Stylesheet resources (e.g., \texttt{.css})
      \item Script resources (e.g., \texttt{.js})
    \end{itemize}
  \item A logging or indexing surrogate is available to record fetched URLs.
\end{itemize}

\subsection*{Test Steps}
\begin{enumerate}
  \item Start the crawler with the configured seed URL.
  \item Allow the crawler to process all reachable links from the seed page.
  \item Inspect the recorded fetch attempts in the log or index surrogate.
\end{enumerate}

\subsection*{Expected Results}
\begin{itemize}
  \item The crawler fetches and processes only HTML documents.
  \item No fetch attempts are made for image, CSS, or JavaScript resources.
  \item Non-relevant resource URLs do not appear in the crawl queue or index.
  \item The crawling process completes without errors.
\end{itemize}

% ==========================================================

% ==========================================================
\section{Test Case: TC-FRQ-1009}

\subsection*{Related Requirements}
FRQ-1009

\subsection*{Description}
Verify that the crawler detects hyperlinks that point to PDF files and
adds those PDF documents to the index so that they become searchable.

\subsection*{Preconditions}
\begin{itemize}
  \item Crawler component is operational
  \item Indexing component is operational
  \item Crawling of external resources (PDF files) is permitted
  \item At least one HTML page exists that contains a hyperlink to a PDF file
  \item The search index is initially empty.
\end{itemize}

\subsection*{Test Steps}
\begin{enumerate}
  \item Configure the crawler with the HTML page as the seed URL.
  \item Start the crawling and indexing process.
  \item Allow the crawler to fetch the HTML page.
  \item Detect and follow the hyperlink to the linked PDF file.
  \item Allow the indexing component to process the discovered PDF file.
  \item Inspect the contents of the search index.
\end{enumerate}

\subsection*{Expected Results}
\begin{itemize}
  \item The crawler detects the hyperlink pointing to the PDF document.
  \item The PDF document is scheduled for crawling or downloading.
  \item The PDF document is added to the index.
  \item Terms extracted from the PDF document become searchable.
  \item No errors occur as a result of handling the PDF file.
\end{itemize}


% ==========================================================
\section{Test Case: TC-FRQ-1010}

\subsection*{Related Requirements}
FRQ-1010

\subsection*{Description}
Verify that the crawler does not extract or follow hyperlinks contained
inside PDF documents. Hyperlinks embedded in PDF files must not be added
to the crawl frontier or visited, even if the crawler downloads the PDF file.

\subsection*{Preconditions}
\begin{itemize}
  \item Crawler component is operational
  \item Crawling of normal HTML hyperlinks is enabled
  \item A reachable PDF document exists that contains one or more hyperlinks
  \item Logging of visited and scheduled URLs is enabled
\end{itemize}

\subsection*{Test Steps}
\begin{enumerate}
  \item Configure a seed URL that links to a PDF document.
  \item Start the crawler with default link-following enabled.
  \item Allow the crawler to access and download the PDF document.
  \item Inspect the crawl frontier (scheduled URL list).
  \item Inspect the list of visited URLs and crawler logs.
\end{enumerate}

\subsection*{Expected Results}
\begin{itemize}
  \item The crawler may download or access the PDF file itself.
  \item No hyperlinks embedded inside the PDF are extracted.
  \item No hyperlinks embedded inside the PDF are scheduled for crawling.
  \item No hyperlinks embedded inside the PDF are visited.
  \item The crawler continues processing other allowed links normally.
  \item No errors or crashes occur as a result of encountering PDF links.
\end{itemize}

\subsection*{Expected Results}
\begin{itemize}
  \item Only visible textual content is extracted.
  \item Images, videos, scripts, and binary resources are ignored.
  \item No non-textual data is stored in the search index.
  \item Extracted text is suitable for term-based indexing.
\end{itemize}
% ==========================================================

% ==========================================================
\section{Test Case: TC-FRQ-2001}

\subsection*{Related Requirements}
FRQ-2001, FRQ-2004

\subsection*{Description}
Verify that the system extracts only textual content from HTML pages and ignores
all non-textual content during the indexing process.

\subsection*{Preconditions}
\begin{itemize}
  \item An HTML page containing visible text content
  \item The same page contains images, videos, scripts, and binary files
  \item Search index is empty
\end{itemize}
\subsection*{Test Steps}
\begin{enumerate}
  \item Submit the HTML page to the indexing component.
  \item Parse the HTML document.
  \item Extract all indexable content.
  \item Inspect extracted data before storage.
  \item Store extracted content in the search index.
\end{enumerate}

\subsection*{Expected Results}
\begin{itemize}
  \item Only visible textual content is extracted.
  \item Images, videos, scripts, and binary resources are ignored.
  \item No non-textual data is stored in the search index.
  \item Extracted text is suitable for term-based indexing.
\end{itemize}
% ==========================================================
\section{Test Case: TC-FRQ-2002}

\subsection*{Related Requirements}
FRQ-2002

\subsection*{Description}
Verify that indexed terms are stored together with references to the pages
in which they appear using an inverted index structure.

\subsection*{Preconditions}
\begin{itemize}
  \item Two or more HTML pages containing overlapping keywords
  \item Search index is empty
\end{itemize}
\subsection*{Test Steps}
\begin{enumerate}
  \item Submit all pages to the indexing component.
  \item Extract and tokenize textual terms from each page.
  \item Normalize extracted terms.
  \item Store terms in the search index.
  \item Inspect the internal index structure.
\end{enumerate}

\subsection*{Expected Results}
\begin{itemize}
  \item Each unique term is stored exactly once in the index.
  \item Each term maps to one or more page references (URLs or document IDs).
  \item Pages containing the same term are associated with that term.
  \item The index follows the inverted index model.
\end{itemize}
% ==========================================================
\section{Test Case: TC-FRQ-2003}

\subsection*{Related Requirements}
FRQ-2003

\subsection*{Description}
Verify that the system supports incremental updates of the search index
without rebuilding the entire index.

\subsection*{Preconditions}
\begin{itemize}
  \item Existing search index populated with indexed pages
  \item One existing page is modified or a new page is added
\end{itemize}
\subsection*{Test Steps}
\begin{enumerate}
  \item Run the indexing process on the initial dataset.
  \item Modify an existing page or add a new page.
  \item Run the indexing process again.
  \item Monitor which pages are re-indexed.
  \item Compare the index state before and after the update.
\end{enumerate}

\subsection*{Expected Results}
\begin{itemize}
  \item Only new or modified pages are re-indexed.
  \item Unchanged indexed pages remain untouched.
  \item No full index rebuild occurs.
  \item The index remains consistent and searchable after the update.
\end{itemize}

% ==========================================================
\section{Test Case: TC-FRQ-3001}

\subsection*{Related Requirements}
FRQ-3001

\subsection*{Description}
Verify that the search engine accepts and correctly processes search queries composed of simple terms and Boolean operators (e.g., AND).

\subsection*{Preconditions}
\begin{itemize}
  \item The search system is running and accessible.
  \item The search index is populated with test data containing known terms.
\begin{itemize}
        \item[] \textit{Example setup:}
        \begin{itemize}
            \item Document A contains only "cats".
            \item Document B contains only "dogs".
            \item Document C contains both "cats" and "dogs".
        \end{itemize}
    \end{itemize}
\end{itemize}

\subsection*{Test Steps}
\begin{enumerate}
  \item Simple Term Search: Submit a search query with a single term: cats.
  \item Multiple Terms Search: Submit a search query with multiple terms: cats dogs.
  \item Operator Search: Submit a search query using the AND operator: cats AND dogs.
\end{enumerate}

\subsection*{Expected Results}
\begin{itemize}
  \item Simple Term: The system returns Document A and Document C (all docs containing "cats").
  \item Multiple Terms: The system returns Document A, B, and C (depending on default ranking, but all relevant docs should be found).
  \item Operator Search: The system returns only Document C (the document containing both terms). This confirms the AND operator is working logically and not just finding any document with either word.
\end{itemize}
% ==========================================================
\section{Test Case: TC-FRQ-3002}

\subsection*{Related Requirements}
FRQ-3002

\subsection*{Description}
Verify that the search engine correctly handles and retrieves results for a query consisting of a single, standalone word.

\subsection*{Preconditions}
\begin{itemize}
        \item The search system is running and the index is populated.
        \item[] \textit{Example setup:}
        \begin{itemize}
            \item Document A contains the word "test".
            \item Document B contains the word "hello".
            \item Document C contains "testing" (to verify exact match if applicable, or just distinct words).
        \end{itemize}
    \end{itemize}

\subsection*{Test Steps}
\begin{enumerate}
        \item Enter the single term \texttt{hello} into the search input.
        \item Execute the search.
        \item Clear results and enter the term \texttt{test}.
        \item Execute the search.
    \end{enumerate}


\subsection*{Expected Results}
\begin{itemize}
        \item For the query \texttt{hello}: Only Document B is returned.
        \item For the query \texttt{test}: Only Document A is returned.
       \item \textbf{Note:} Document C ("testing") should \textbf{not} be returned for the query "test", verifying that the search handles exact word boundaries correctly.
    \end{itemize}

\end{document}
