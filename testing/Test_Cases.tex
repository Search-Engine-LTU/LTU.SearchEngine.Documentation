\documentclass{article}

% --------------------
% Encoding & font
% --------------------
\usepackage[utf8]{inputenc}
\usepackage[T1]{fontenc}
\usepackage{lmodern}

% --------------------
% Page layout
% --------------------
\usepackage[a4paper, margin=2cm]{geometry}

% --------------------
% Title info
% --------------------
\title{Test Case Specification}
\author{Group 1}
\date{\today}

\begin{document}

\maketitle

\section{Introduction}
This document describes test cases for the LTU Search Engine project.
% ==========================================================
\section{Test Case: TC-FRQ-1009}

\subsection*{Related Requirements}
FRQ-1009

\subsection*{Description}
Verify that the crawler detects hyperlinks that point to PDF files and
adds those PDF documents to the index so that they become searchable.

\subsection*{Preconditions}
\begin{itemize}
  \item Crawler component is operational
  \item Indexing component is operational
  \item Crawling of external resources (PDF files) is permitted
  \item At least one HTML page exists that contains a hyperlink to a PDF file
  \item The search index is initially empty
\end{itemize}

\subsection*{Test Steps}
\begin{enumerate}
  \item Configure the crawler with the HTML page as the seed URL.
  \item Start the crawling and indexing process.
  \item Allow the crawler to fetch the HTML page.
  \item Detect and follow the hyperlink to the linked PDF file.
  \item Allow the indexing component to process the discovered PDF file.
  \item Inspect the contents of the search index.
\end{enumerate}

\subsection*{Expected Results}
\begin{itemize}
  \item The crawler detects the hyperlink pointing to the PDF document.
  \item The PDF document is scheduled for crawling or downloading.
  \item The PDF document is added to the index.
  \item Terms extracted from the PDF document become searchable.
  \item No errors occur as a result of handling the PDF file.
\end{itemize}


% ==========================================================
\section{Test Case: TC-FRQ-2001}

\subsection*{Related Requirements}
FRQ-2001, FRQ-2004

\subsection*{Description}
Verify that the system extracts only textual content from HTML pages and ignores
all non-textual content during the indexing process.

\subsection*{Preconditions}
\begin{itemize}
  \item An HTML page containing visible text content
  \item The same page contains images, videos, scripts, and binary files
  \item Search index is empty
\end{itemize}
\subsection*{Test Steps}
\begin{enumerate}
  \item Submit the HTML page to the indexing component.
  \item Parse the HTML document.
  \item Extract all indexable content.
  \item Inspect extracted data before storage.
  \item Store extracted content in the search index.
\end{enumerate}

\subsection*{Expected Results}
\begin{itemize}
  \item Only visible textual content is extracted.
  \item Images, videos, scripts, and binary resources are ignored.
  \item No non-textual data is stored in the search index.
  \item Extracted text is suitable for term-based indexing.
\end{itemize}
% ==========================================================
\section{Test Case: TC-FRQ-2002}

\subsection*{Related Requirements}
FRQ-2002

\subsection*{Description}
Verify that indexed terms are stored together with references to the pages
in which they appear using an inverted index structure.

\subsection*{Preconditions}
\begin{itemize}
  \item Two or more HTML pages containing overlapping keywords
  \item Search index is empty
\end{itemize}
\subsection*{Test Steps}
\begin{enumerate}
  \item Submit all pages to the indexing component.
  \item Extract and tokenize textual terms from each page.
  \item Normalize extracted terms.
  \item Store terms in the search index.
  \item Inspect the internal index structure.
\end{enumerate}

\subsection*{Expected Results}
\begin{itemize}
  \item Each unique term is stored exactly once in the index.
  \item Each term maps to one or more page references (URLs or document IDs).
  \item Pages containing the same term are associated with that term.
  \item The index follows the inverted index model.
\end{itemize}
% ==========================================================
\section{Test Case: TC-FRQ-2003}

\subsection*{Related Requirements}
FRQ-2003

\subsection*{Description}
Verify that the system supports incremental updates of the search index
without rebuilding the entire index.

\subsection*{Preconditions}
\begin{itemize}
  \item Existing search index populated with indexed pages
  \item One existing page is modified or a new page is added
\end{itemize}
\subsection*{Test Steps}
\begin{enumerate}
  \item Run the indexing process on the initial dataset.
  \item Modify an existing page or add a new page.
  \item Run the indexing process again.
  \item Monitor which pages are re-indexed.
  \item Compare the index state before and after the update.
\end{enumerate}

\subsection*{Expected Results}
\begin{itemize}
  \item Only new or modified pages are re-indexed.
  \item Unchanged indexed pages remain untouched.
  \item No full index rebuild occurs.
  \item The index remains consistent and searchable after the update.
\end{itemize}

\end{document}
