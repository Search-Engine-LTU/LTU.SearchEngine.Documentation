\section{Search Engine Project – Group Working Guidelines}
\label{sec:group-working-guidelines}

\subsection{Purpose}

These guidelines aim to support a sustainable, respectful, and collaborative way of working throughout the project.
They are intended as shared support for the group and can be revisited or adjusted if needed.

\subsection{Communication}

The primary communication channel for the project is Microsoft Teams.

Project-related discussions, decisions, and updates are handled in Teams unless otherwise agreed.

Meetings are held in Teams unless explicitly stated otherwise.

\subsection{Availability \& Absence}

Planned or unplanned absence is communicated in Teams as soon as possible.

We assume good intent and respect that availability may vary over time.

\subsection{Time Expectations}

This is a 6 hp course at 40\% pace.

The expected workload is approximately 16 hours per week, or 3--3.5 hours per weekday.

Work is primarily done on weekdays.

No one is expected to work evenings or weekends unless explicitly and mutually agreed upon.

\subsection{Sprint Structure and Planning}

The project is organized using short, time-boxed sprints to support clarity, focus, and sustainability.

\begin{itemize}
  \item A sprint has a duration of \textbf{one week}, running from \textbf{Monday to Monday}.
  \item Sprint planning takes place on \textbf{Mondays}.
  \item Each sprint should have a clear sprint goal aligned with the current milestone.
  \item Work planned for a sprint should reflect realistic capacity and agreed priorities.
\end{itemize}

\subsection{Daily Stand-ups}

Daily stand-up meetings are used for short synchronization and status updates.

\begin{itemize}
  \item Daily stand-ups are scheduled at \textbf{09:30} on days without lectures.
  \item On days with scheduled lectures, the daily stand-up may be held later in the day.
  \item Stand-ups are intended to be brief and focused on coordination rather than problem-solving.
\end{itemize}

\subsection{Tempo \& Sustainability}

We aim for a sustainable project tempo aligned with the course workload (6 hp, 40\%).

Individual differences in availability should not translate into pressure or expectations on others.

Progress should support shared understanding and collaboration, rather than rapid solo advancement.

\subsection{Collaboration \& Workflow}

Tasks that affect the group should be discussed before implementation.

We aim to avoid solo implementations of major features that impact others.

Agreed workflows (e.g.\ branches, pull requests, reviews) should be followed to ensure transparency and inclusion.

\subsection{Agile Principles}

We value:
\begin{itemize}
  \item shared understanding
  \item incremental progress
  \item collaboration over individual output
\end{itemize}

Progress should be visible and accessible to the whole group.

\subsection{Feedback \& Tone}

Feedback should be:
\begin{itemize}
  \item constructive
  \item respectful
  \item focused on the work, not the person
\end{itemize}

When possible, individual feedback is given privately rather than in group channels.

\subsection{Decision-Making}

Decisions that affect the whole group are discussed and agreed upon together.

The goal is alignment and inclusion, not speed at the expense of collaboration.

\subsection{Shared Responsibility}

Everyone is responsible for their assigned tasks.

No one is expected to compensate for others by overworking.

We respect that different people contribute in different ways within the agreed scope.

\bigskip
These guidelines are meant as support, not control, and exist to help the group work effectively and sustainably throughout the project.
