\documentclass[a4paper,11pt]{article}

% --- Settings ---
\usepackage[utf8]{inputenc}
\usepackage[T1]{fontenc}
\usepackage[english]{babel} % English language
\usepackage[margin=2.5cm]{geometry}
\usepackage{parskip}     % Nice spacing between paragraphs
\usepackage{enumitem}    % For compact lists
\usepackage{amssymb}

% --- Main Title (Cover Page info) ---
\title{\textbf{Meeting Notes}}
\author{Project: LTU Search Engineee}
\date{} % Keeps the date empty on the title page

\begin{document}

\maketitle
% ==========================================
%           MEETING START
% ==========================================

\section*{Meeting: 2026-01-13 (Daily Stand-up)}

\textbf{Attendees:} \\
Cilla, Nattarintra, Jean-Paul, Emma, Linnea

\subsection*{Discussion and Notes}
A daily stand-up meeting was held where each team member briefly reported on what they worked on previously and what they are currently working on.

\begin{itemize}
    \item \textbf{Cilla} was absent the previous day due to illness. She will continue working on the test case she started last week.
    \item \textbf{Nattarintra} continued working on test cases FRQ-3004 and FRQ-3006 and has conducted research on Redis and Celery. She will continue this research today.
    \item \textbf{Jean-Paul} has researched Redis and Celery, created several pull requests, and reviewed pull requests from other team members. He would like to further discuss ideas related to Redis and Celery and how the team should proceed.
    \item \textbf{Emma} has continued working on pull requests and has written review comments with suggestions for improvements. She will continue reviewing the remaining pull requests.
    \item \textbf{Linnea} has finalized and documented meeting notes from the previous meeting and created a pull request for these notes. She has also reviewed all currently open pull requests and created a prioritization table, available in OneNote under the tab \textit{``Needing to be researched''}. Linnea will assign herself to one or more pull requests for review and will also research Redis and Celery to gain a better overall understanding.
\end{itemize}

The team discussed whether an additional role, such as a Lead Developer, was needed at this stage. It was agreed that this role is not required at the moment, but the topic will be revisited later if the need arises as development progresses.

\subsection*{Focus and Next Steps}
The main focus for the week remains on the crawler and the 1000-series requirements.

\subsection*{Next Meeting}
The next meeting will be held on January 14th at 08:30 as a daily stand-up.

% ==========================================
%           MEETING END
% ==========================================

\newpage
% ==========================================
%           MEETING START
% ==========================================

\section*{Meeting: 2026-01-12, 13:10}

\textbf{Attendees:} \\
Linnea, Emma, Jean-Paul, Nattarintra

\subsection*{Context}
The meeting focused on synchronizing current work, confirming responsibilities for the coming days, and establishing a shared prioritization model for GitHub issues and pull requests.

\subsection*{Current Work Status}
Each team member briefly reported on their current tasks and plans for the day:
\begin{itemize}
    \item \textbf{Jean-Paul} continues working on the UML and class diagrams.
    \item \textbf{Emma} and \textbf{Nattarintra} continue working on the GitHub tasks they started the previous week.
    \item \textbf{Linnea} will review all pending pull requests and establish a priority order for which PRs must be approved before Thursday.
\end{itemize}

Once a pull request is considered ready for approval, team members may self-assign and proceed with review and merge.

\subsection*{Prioritization Model}
The group agreed on a shared priority model for GitHub issues and tasks:
\begin{itemize}
    \item \textbf{P0:} High priority
    \item \textbf{P1:} Medium priority
    \item \textbf{P2:} Low priority
\end{itemize}

When creating a new issue on GitHub, a priority level should be assigned. Priorities may be adjusted as the work progresses.

\subsection*{Next Meeting}
\begin{itemize}
    \item \textbf{Daily Standup:} Tuesday, January 13 at 13:00
    \item Duration: Maximum 15 minutes
\end{itemize}

% ==========================================
%           MEETING END
% ==========================================

\newpage

% ==========================================
%           MEETING START
% ==========================================

\section*{Meeting: 2026-01-09}

\textbf{Attendees:} \\
Jean-Paul, Emma, Nattarintra, Linnea, Cilla

\subsection*{Context}
This was a short follow-up meeting held after the supervisor meeting, with the purpose of synchronizing roles and tasks ahead of the next supervisor meeting.

\subsection*{Key Deadlines}
\begin{itemize}
    \item \textbf{UML and Class Diagrams:} Must be completed before the next supervisor meeting on \textbf{Thursday, January 15 at 15:00}.
\end{itemize}

\subsection*{Roles and Responsibilities}
The following roles and task distribution were agreed upon:

\begin{itemize}
    \item \textbf{Project Lead: Linnea}
    \begin{itemize}
        \item Review pending Pull Requests.
        \item Create a priority list identifying PRs that block project progress.
        \item Highlight crawler-related PRs as highest priority.
        \item Team members will self-assign PRs for review and approval.
    \end{itemize}

    \item \textbf{UML and Class Diagram:}
    \begin{itemize}
        \item Jean-Paul and Cilla will work on the UML Class Diagram for the crawler.
    \end{itemize}

    \item \textbf{GitHub Tasks:}
    \begin{itemize}
        \item Nattarintra and Emma will continue working on tasks and issues available in the GitHub repository.
    \end{itemize}
\end{itemize}

It was agreed that roles and responsibilities will rotate between group members to ensure shared understanding and balanced workload.

\subsection*{Testing Strategy}
The group discussed test case design and agreed on the following principles:
\begin{itemize}
    \item Test cases should be as focused and clean as possible.
    \item Each test case should verify \textbf{one specific behavior or requirement}.
    \item The goal is to enable clear and maintainable unit testing in the implementation.
\end{itemize}

\subsection*{Technical Discussions}
\begin{itemize}
    \item The group agreed to investigate suitable frameworks or approaches for handling \textbf{normalized words}.
    \item Jean-Paul reviewed the ER diagram to ensure that all group members had a clear understanding of the data model.
\end{itemize}

% ==========================================
%           MEETING END
% ==========================================
\newpage
% ==========================================
%        SUPERVISOR MEETING START
% ==========================================

\section*{Supervisor Meeting: 2026-01-08}

\textbf{Supervisor:} Malte Kerl \\
\textbf{Attendees:} Emma, Cilla, Jean-Paul, Nattarintra, Linnea \\

\subsection*{Feedback on ER Diagram}

\begin{itemize}
  \item A question was raised regarding how the system represents pages that are no longer available.
  One possible solution is to simply remove such pages from the database. The solution does not need to be sophisticated, but the group should be able to reason about and motivate the chosen approach.

  \item The use of a \textbf{normalized representation of words} was recommended. At the moment, several different representations of the same word may exist. The purpose and benefit of normalization should be clearly understood and applied.

  \item Storing a \textbf{datetime for last crawled} was confirmed to be a good design choice.

  \item Clarification was requested regarding the purpose of \textbf{HTTP status}.
  If it is used to indicate removed or redirected pages (e.g.\ HTTP 302), then it is reasonable to keep it.

  \item The attribute \texttt{priority\_integer} can be removed for now, as it is not necessary.
  Similarly, the \texttt{Done} status in the crawl queue can be removed.

  \item Overall, the ER diagram is considered largely complete.

  \item For handling concurrency in the crawler, tools such as \textbf{Redis} or \textbf{Celery} may be used (links were provided by Malte).
  The group is encouraged to attempt implementing concurrency. However, if it turns out to be too complex, it is acceptable to skip it.
\end{itemize}

\subsection*{Module Diagram and Architecture}

\begin{itemize}
  \item The module diagram is generally acceptable.
  \item Authentication should be removed.
  \item The paginator should be moved either to the database layer or to \texttt{Backend.Application}.
  \item The role and necessity of a \texttt{BackgroundService} should be reconsidered and clearly motivated.
\end{itemize}

\subsection*{Process and Testing}

\begin{itemize}
  \item The documentation repository and the current use of GitHub issues were reviewed and considered appropriate.

  \item Malte will check with Ulf whether the project should follow SCRUM, and if a backlog is required.
  It may be useful to define a few initial use cases at the beginning, until development of the search engine is underway.

  \item The project does not need to strictly follow Test Driven Development (TDD).
  However, testing is required, and tests should be written continuously alongside development, not postponed until after implementation.
\end{itemize}

\subsection*{Design and Readiness}

\begin{itemize}
  \item The module diagram should be used as the basis for completing the UML and class diagrams.

  \item Malte asked whether the group feels ready to start parallel coding.
  The group expressed that this is not yet the case, as parts of the overall structure and "big picture" are still unclear.
  This is expected to become clearer once the diagrams are finalized.

  \item The query syntax must be defined and documented.

  \item Contact information pages should be crawled at this stage.
  In the future, a solution for overriding \texttt{robots.txt} for such pages will be needed.
\end{itemize}

\subsection*{Next Steps}

\begin{itemize}
  \item Finalize UML and class diagrams and send them to Malte.
  \item Prepare for the next supervisor meeting on \textbf{Thursday, January 15 at 10:00}.
\end{itemize}

% ==========================================
%        SUPERVISOR MEETING END
% ==========================================

\newpage

% ==========================================
%           MEETING START
% ==========================================

\section*{Meeting: 2026-01-08}

\textbf{Attendees:} Cilla, Emma, Jean-Paul, Nattarintra, Linnea \\

\subsection*{Topics Discussed}

\begin{itemize}
  \item The group discussed questions to be raised with the supervisor regarding:
  \begin{itemize}
    \item Whether the project should follow Test Driven Development (TDD).
    \item If TDD would imply additional workload by writing tests before implementation and then iterating.
    \item The need for clarification on what TDD means in practice for this project.
    \item Whether contact information pages should be included in the crawling scope.
    \item With respect to FRQ-1003, whether the crawler should support concurrent crawling.
    \item If concurrency increases complexity or execution time.
    \item The consequences of not using concurrency.
  \end{itemize}

  \item It was noted that the group should organise the work in a more structured way, including clearer role and task distribution.

  \item The documentation is not yet complete. There are still test cases that need to be written for several requirements.

  \item On GitHub, issues exist for most backend-related tasks, while no issues have yet been created for frontend work.
\end{itemize}

\subsection*{Task Distribution}

\begin{itemize}
  \item Jean-Paul will continue writing test cases for crawler-related requirements (RQ 1000-series).
  \item Cilla has also started working on crawler test cases and will continue contributing in that area.
  \item Nattarintra will begin working on requirements and test cases in the RQ 3000-series.
  \item Emma, Cilla, and Linnea will conduct a short walkthrough of the development environment in VS Code to investigate and resolve issues encountered when running the program.
\end{itemize}

\subsection*{Next Steps}

\begin{itemize}
  \item Model the crawler and incorporate feedback from the supervisor meeting later the same day (15:00).
  \item Work towards clearer role definitions and task distribution within the group.
\end{itemize}

% ==========================================
%           MEETING END
% ==========================================

\newpage

% ==========================================
%           MEETING START
% ==========================================

\section*{Meeting: 2026-01-05} % <--- Change Date Here

\textbf{Attendees:} \\
Emma, Cilla, Jean-Paul, Nattarintra

\subsection*{Agenda}
\begin{enumerate}[noitemsep]
    \item What was done during the holidays
    \item Divide work between members.
\end{enumerate}

\subsection*{Discussion and Notes}
We discussed what we have been doing during holidays and how we should continue with the project.


\subsection*{Action Items}
\begin{itemize}
    \item \textbf{JP:} Summary of diagram 
    \item \textbf{Emma:} Write testcase requirements
    \item \textbf{Cilla:} Write testcase requirements
    \item \textbf{Nattarintra:} Write testcase requirements
\end{itemize}

% ==========================================
%           MEETING END
% ==========================================

\newpage

% ==========================================
%           MEETING START
% ==========================================

\section*{Meeting: 2025-12-19} % <--- Change Date Here

\textbf{Attendees:} \\
Emma, Linnea, Jean-Paul

\subsection*{Agenda}
\begin{enumerate}[noitemsep]
    \item What was done the day before.
    \item Discuss reporting.
    \item Divide work between members.
\end{enumerate}

\subsection*{Discussion and Notes}
Everyone present mentioned what they have done they day before.
We discussed reporting and decided to send our current activity report in conjunction with a small "to do next week" list.


\subsection*{Action Items}
\begin{itemize}
    \item \textbf{JP:} Review meeting notes PR, Tests
    \item \textbf{Emma:} What to include in an index, Review Query syntax PR
    \item \textbf{Linnea:} How to limit crawler rate limit
    \item \textbf{Cilla:} Investigate different algorithms for searching
    \item \textbf{Nattarintra:} Test
\end{itemize}

% ==========================================
%           MEETING END
% ==========================================

\newpage

% ==========================================
%           MEETING START
% ==========================================

\section*{Meeting: 2025-12-18} % <--- Change Date Here

\textbf{Attendees:} \\
Emma, Cilla, Linnea, Jean-Paul, Nattarintra

\subsection*{Agenda}
\begin{enumerate}[noitemsep]
    \item Divide work between members
\end{enumerate}

\subsection*{Discussion and Notes}
We focused on dividing the tasks for the search engine project.
Everyone has been assigned a specific area to research or implement.

\subsection*{Action Items}
\begin{itemize}
    \item \textbf{JP:} Query syntax
    \item \textbf{Emma:} What to include in an index, upload meetingnotes to Git Repository
    \item \textbf{Linnea:} How to limit crawler rate limit
    \item \textbf{Cilla:} Investigate different algorithms for searching
    \item \textbf{Nattarintra:} Test
\end{itemize}

% ==========================================
%           MEETING END
% ==========================================

\newpage

% ==========================================
%           MEETING START
% ==========================================

\section*{Meeting: 2025-12-17}
\textbf{Context:} Updated directives from Malte (replaces parts of original spec).

\textbf{Attendees:} \\
Emma, Cilla, Linnea, Jean-Paul, Nattarintra

\subsection*{1. Technical Requirements (Scope)}
\begin{itemize}
    \item \textbf{Database:} 
    \begin{itemize}
        \item Free choice (Postgres, MariaDB, SQLite are all okay).
        \item \textit{Important:} Avoid MS SQL Server (due to Linux compatibility issues). Choose something easy to install/run.
        \item Storage: 512 GB is available on the VM, but we likely only need a few hundred MB.
    \end{itemize}
    
    \item \textbf{Web Crawler Behavior:}
    \begin{itemize}
        \item \textbf{Whitelist:} Strict. Only crawl \texttt{ltu.se} and specific links provided. Do \textbf{not} follow external links (e.g., Facebook, Instagram).
        \item \textbf{Robots.txt:} Generally respect it, \textit{unless} it blocks "contacts" or "users". We must index contact pages so staff can be found.
        \item \textbf{Content:} Skip images to reduce complexity. Focus on text.
        \item \textbf{Rate Limiting:} No hard limit, but "don't DDoS". A few hundred requests per minute is acceptable.
    \end{itemize}

    \item \textbf{Search Algorithm \& Ranking:}
    \begin{itemize}
        \item Use a combination of \textbf{PageRank} (link popularity) and \textbf{Word Frequency}.
        \item We are encouraged to experiment with algorithms to find the best results.
    \end{itemize}
\end{itemize}

\subsection*{2. Documentation \& Requirements (Milestone 0)}
\begin{itemize}
    \item \textbf{Requirements Document:} 
    \begin{itemize}
        \item Add a \textbf{third column} for "Test Cases". (Be specific: How do we prove the requirement is met?)
        \item Remove "Test Coverage" from the requirements doc (keep it in code, but not as a functional requirement).
    \end{itemize}
    \item \textbf{Diagrams:} Create initial drafts for UML Diagrams (e.g. a Class Diagram), ER-Diagram (Database). Share these early ("Share everything").
    \item \textbf{Tools:} Documentation in LaTeX, Code in Git.
\end{itemize}

\subsection*{3. Project Structure \& Team Roles}
Total team size: 5 members.
\begin{itemize}
    \item \textbf{1 Lead Developer} (Responsible for overall architecture).
    \item \textbf{2 Frontend Developers}.
    \item \textbf{2 Backend Developers}.
\end{itemize}
\textbf{Workflow:} Use Code Reviews and Pull Requests (PR). Clear separation of tasks between Frontend and Backend.

\subsection*{4. Reporting \& Next Steps}
\begin{itemize}
    \item \textbf{Weekly Report:} Send an email \textbf{every Friday}.
    \begin{itemize}
        \item Content: "Activity report" (what we did) and "Next to-dos" (what we will do).
        \item \textit{Note:} Send one immediately before the holidays.
    \end{itemize}
    \item \textbf{Next Meeting:} January 8th at 15:00.
\end{itemize}

% ==========================================
%           MEETING END
% ==========================================

\newpage
% ==========================================
%           MEETING START
% ==========================================

\section*{Meeting: 2025-12-16} % <--- Change Date Here

\textbf{Attendees:} \\
Emma, Cilla, Linnea, Jean-Paul, Nattarintra

\subsection*{Agenda}
\begin{enumerate}[noitemsep]
    \item Meeting for preparation 
\end{enumerate}

\subsection*{Discussion and Notes}
We had a shorter meeting to prepare questions regarding tomorrows meeting with Malte.

% ==========================================
%           MEETING END
% ==========================================

\newpage

% ==========================================
%           MEETING START
% ==========================================
\section*{Meeting: 2025-12-15} % <--- Change Date Here

\textbf{Attendees:} \\
Emma, Cilla, Linnea, Jean-Paul, Nattarintra

\subsection*{Agenda}
\begin{enumerate}[noitemsep]
    \item Meeting for preparation 
\end{enumerate}

\subsection*{Discussion and Notes}
We had a shorter meeting to prepare questions regarding tomorrows meeting with Malte.
% ==========================================
%           MEETING END
% ==========================================

\newpage
% ==========================================
%           MEETING START
% ==========================================

\section*{Meeting: 2025-12-11} % <--- Change Date Here

\textbf{Attendees:} \\
Emma, Cilla, Linnea, Jean-Paul, Nattarintra

\subsection*{Agenda}
\begin{enumerate}[noitemsep]
    \item Daily meeting we discussed what to focus on, Jean-Paul set up a Req document and we started to fill it up whit Test and priorities them. 
\end{enumerate}

\subsection*{Discussion and Notes}
We had a shorter meeting to prepare questions regarding tomorrows meeting with Malte.
% ==========================================
%           MEETING END
% ==========================================

\newpage


% ==========================================
%           MEETING START
% ==========================================

\section*{Meeting: 2025-12-10 (Project Kick-off)}
\textbf{Attendees:} Team and Malte (Client)

\subsection*{Agenda}
\begin{enumerate}[noitemsep]
    \item Review of project scope and limitations
    \item Technical requirements clarification
    \item Expectations for Milestone 0
\end{enumerate}

\subsection*{Decisions \& Directives (Summary)}

\subsubsection*{1. Scope \& Crawler Behavior}
\begin{itemize}
    \item \textbf{Whitelist:} Strictly \texttt{*.ltu.se}. Do not follow external links (Facebook, etc.). Future domains like \texttt{islab.se} might be added later.
    \item \textbf{Robots.txt:} strictly adhere to it for Milestone 0. (Exceptions for contact pages might be added later).
    \item \textbf{Dynamic Content:} LTU's dynamic content is server-side rendered, so no "headless browser" is required.
    \item \textbf{PDFs:} Crawler must detect PDF links. Indexing the \textit{content} of PDFs is optional (nice-to-have).
\end{itemize}

\subsubsection*{2. Search Functionality}
\begin{itemize}
    \item \textbf{Queries:} Optimized for simple queries (2--3 words). Ranking priority: All terms $>$ Some terms $>$ One term.
    \item \textbf{Pagination:} Required (e.g., 10, 20 results per page).
    \item \textbf{Performance:} Response time should be under 10 seconds.
    \item \textbf{Optional features:} Wildcards (*), category filters, and boolean operators are \textit{not} mandatory.
\end{itemize}

\subsubsection*{3. Technology \& Deployment}
\begin{itemize}
    \item \textbf{UI:} Plain HTML/CSS/JS is perfectly acceptable. No complex frameworks (React/Vue) needed unless desired.
    \item \textbf{Deployment:} Local execution is enough. No server deployment required yet.
\end{itemize}

\subsection*{Deliverables for Milestone 0}
\begin{itemize}
    \item \textbf{Requirements Document:} Must include Stakeholder requirements, Functional requirements, and Non-functional requirements.
    \item \textbf{Diagrams:} UML Class diagram + ER Diagram (Database).
\end{itemize}

% ==========================================
%           MEETING END
% ==========================================

\newpage

% ==========================================
%           MEETING START
% ==========================================

\section*{Meeting: 2025-12-09} % <--- Change Date Here

\textbf{Attendees:} \\
Emma, Cilla, Linnea, Jean-Paul, Nattarintra

\subsection*{Agenda}
\begin{enumerate}[noitemsep]
    \item Get to know eachother
    \item Send a mail to Malte
    \item Set up a Github project
\end{enumerate}

\subsection*{Discussion and Notes}
We had a presentation round and talked about the project and sent a email to Malte for scheduling a meeting. Finally we set up a Github Project

% ==========================================
%           MEETING END
% ==========================================
\end{document}