\documentclass[a4paper,11pt]{article}

% --- Settings ---
\usepackage[utf8]{inputenc}
\usepackage[T1]{fontenc}
\usepackage[english]{babel} % English language
\usepackage[margin=2.5cm]{geometry}
\usepackage{parskip}     % Nice spacing between paragraphs
\usepackage{enumitem}    % For compact lists
\usepackage{amssymb}

% --- Main Title (Cover Page info) ---
\title{\textbf{Meeting Notes}}
\author{Project: LTU Search Engineee}
\date{} % Keeps the date empty on the title page

\begin{document}

\maketitle

% ==========================================
%           MEETING START
% ==========================================

\section*{Meeting: 2026-01-26 (Daily Stand-up)}

\textbf{Date:} January 26 \\
\textbf{Time:} 13:15 \\
\textbf{Purpose:} Daily synchronization and status updates \\
\textbf{Attendees:} Nattarintra, Jean-Paul, Emma, Cilla, Linnea

\subsection*{Discussion and Notes}
We decided not to have a dedicated project manager; instead, we will all share that responsibility during our daily scrum. We discussed that the focus for the week is to implement a crawler that we can demonstrate on Friday, as well as setting up an index diagram.

Team Division:

Emma and Jean-Paul: Focusing on the crawler

Linnea and Cilla: Focusing on the front end

Nattarintra, Emma, and Cilla: Focusing on producing an index diagram
% ==========================================
%           MEETING END
% ==========================================
\newpage
% ==========================================
%           MEETING START
% ==========================================
\section*{Meeting: 2026-01-22 (Supervisor Meeting)}

\textbf{Time:} 15:00 \\
\textbf{Purpose:} Supervisor meeting – clarification of team structure, architecture feedback, and next steps

\textbf{Attendees:} \\
Jean-Paul, Nattarintra, Emma, Cilla, Linnea, Malte

\subsection*{Discussion and Notes}

The meeting began with a request for clarification regarding the division of work into teams. Malte explained that the proposed structure consists of three teams:
\begin{itemize}
  \item A crawler team
  \item An index team
  \item A query processing and frontend team
\end{itemize}

Based on this structure, the following distribution of team members was agreed upon:
\begin{itemize}
  \item \textbf{Crawler Team:} Jean-Paul (full-time), Emma (part-time)
  \item \textbf{Index Team:} Nattarintra (full-time), Cilla (part-time), Emma (part-time)
  \item \textbf{Frontend and Query Processing Team:} Linnea (full-time), Cilla (part-time)
\end{itemize}

It was noted that this team division is flexible. Group members are expected to support other teams when needed, as workload and intensity may vary over time.

The group received feedback on the updated architecture diagram, which no longer includes a database component. The diagram was considered sufficiently clear and appropriate for beginning implementation work.

For the next supervisor meeting, each team is expected to present:
\begin{itemize}
  \item An updated diagram from their respective perspective
  \item A plan for implementation
  \item Some form of initial implementation, in particular from the crawler team
\end{itemize}

\subsection*{Next Meeting}

The next supervisor meeting with Malte is scheduled for \textbf{Friday, January 30 at 10:00}.

% ==========================================
%           MEETING END
% ==========================================

\newpage


% ==========================================
%           MEETING START
% ==========================================

\section*{Meeting: Synchronization Meeting Before Supervisor Meeting}

\textbf{Date:} January 22 \\
\textbf{Time:} 09:15 \\
\textbf{Purpose:} Internal synchronization ahead of the supervisor meeting \\
\textbf{Attendees:} Nattarintra, Jean-Paul, Emma, Cilla, Linnea

\subsection*{Project Roles}

It was noted that Linnea will continue in the role of project lead until a new project lead is formally appointed.

\subsection*{Group Working Guidelines}

The group reviewed a draft document proposing \textit{Group Working Guidelines}. The purpose of the document is to provide support and structure to help the group work effectively and sustainably throughout the project.

The group decided to keep the document and include it in the documentation repository on GitHub as part of the project’s working documentation.

The following additions and clarifications were agreed upon:
\begin{itemize}
  \item A sprint has a duration of one week, running from Monday to Monday.
  \item Sprint planning will take place on Mondays.
  \item Daily stand-up meetings will be held at 09:30 on days without lectures.
  \item On days with scheduled lectures, daily stand-ups may be held later in the day.
\end{itemize}

\subsection*{Work Distribution for Milestone 1}

The group reviewed a proposed work distribution intended primarily for the start of Milestone 1. While there is room for improvement, the proposal was seen as a useful starting point and a way to concretize which areas should be prioritized during Milestone 1.

No formal role or task assignments were made at this stage. The group agreed to await feedback and clarifications from the supervisor (Malte) before finalizing responsibilities.

\subsection*{Preparation for Supervisor Meeting}

Ahead of the meeting with Malte, all group members were asked to reflect on potential questions. These questions should be posted in the Teams chat so that Linnea can compile them into a single document prior to the supervisor meeting.

\subsection*{Current Status and Milestone 0}

An assessment of the current project status indicates that Milestone 0 is not yet fully completed. Test cases still need to be written across all series.

The group will aim to complete the remaining test cases during the remainder of the current week, with the goal of finishing Milestone 0 by Monday.
% ==========================================
%           MEETING END
% ==========================================
\newpage
% ==========================================
%           MEETING START
% ==========================================
\section*{Meeting: 2026-01-19 (Daily Stand-up)}

\textbf{Time:} 09:20 \\
\textbf{Purpose:} Daily synchronization and status updates

\textbf{Attendees:} \\
Emma, Jean-Paul, Nattarintra

\subsection*{Discussion and Notes}
A short daily stand-up meeting where we discussed how to continue.

Jean-Paul has been working on a new diagram for the crawlern but is unsure how TPL Queue works, it is divided into modules but how do they work together.
Emma has been studying the diagram and will research how modules works in TPL and describe the diagram.
Nattarintra continue doing test cases for the 3000 series.

% ==========================================
%           MEETING END
% ==========================================


% ==========================================
%           MEETING START
% ==========================================

\section*{Meeting: 2026-01-16 (Daily Stand-up)}

\textbf{Time:} 09:00 \\
\textbf{Purpose:} Daily synchronization and status updates

\textbf{Attendees:} \\
Cilla, Emma, Jean-Paul, Nattarintra, Linnea

\subsection*{Discussion and Notes}

The meeting was held as a daily stand-up with a short round of status updates from each participant.

Cilla informed the group that she was unavailable during the previous afternoon. Today, her primary focus will be on the course \textit{Computer and Network Security}, as preparation is required for an upcoming quiz.

Emma reported that she reviewed the updated UML diagram created by Jean-Paul the previous day. Today, she will continue reviewing the diagram and also spend time studying for the \textit{Computer and Network Security} course.

Jean-Paul explained that he updated the UML diagram in accordance with feedback received from the supervisor, Malte. The updated diagram now reflects an in-memory solution instead of a database. For the remainder of the day, he will focus on work related to the other course.

Nattarintra reported that she reviewed a pull request previously commented on by Linnea, which contained merge conflicts. 
There seemed to be some kind of issue in Github regarding branches and tickets. The group discussed the issue and identified that the problem was caused by a branch not being created directly from the corresponding GitHub issue. The issue was considered minor and is expected to be avoided going forward. Today, Nattarintra will prepare for an upcoming internship interview with a company in Umeå.

Linnea reported that she cleaned up and published meeting minutes from three meetings held the previous day. She also merged one pull request and reviewed another. Today, remaining meeting notes will be cleaned up and published. Linnea will also begin planning a smooth handover to the next week’s project lead and secretary. If time allows, a high-level review of the project will be conducted to improve overall overview and alignment.

\subsection*{Decisions and Next Steps}

It was noted that the decision regarding the next \textbf{Project Lead} and \textbf{Secretary} will be made during the meeting on Monday.

% ==========================================
%           MEETING END
% ==========================================
\newpage



% ==========================================
%           MEETING START
% ==========================================

\section*{Meeting: 2026-01-15 (Post-Supervisor Sync Meeting)}

\textbf{Time:} 11:05 \\
\textbf{Purpose:} Follow-up and resynchronization after the supervisor meeting

\textbf{Attendees:} \\
Cilla, Emma, Jean-Paul, Nattarintra, Linnea

\subsection*{Discussion and Notes}

The meeting focused on aligning the group on next steps based on the outcomes and recommendations from the supervisor meeting with Malte.

The group discussed what needs to be completed in order to move forward in the project. It was agreed that the primary focus at this stage is to finalize all requirements and test cases in the \textbf{1000-series (Crawler)}.

Additionally, the UML diagram needs to be revised and simplified in accordance with the approach suggested during the supervisor meeting. The revised UML should reflect a clearer and less complex solution that is easier for the entire group to understand and build upon.

Once the crawler requirements, associated test cases, and the updated UML diagram are completed, the group will have fulfilled the goals of \textbf{Milestone 0}. At that point, the project will be ready to move into the next phase.

It was also noted that discussions regarding detailed work distribution and responsibility assignment will take place after Milestone 0 has been completed.

\subsection*{Next Meeting}
The next meeting will be a daily stand-up scheduled for \textbf{Friday, January 16 at 09:00}.

% ==========================================
%           MEETING END
% ==========================================
\newpage
% ==========================================
%           MEETING START
% ==========================================

\section*{Meeting: 2026-01-15 (Supervisor Meeting)}

\textbf{Time:} 10:00 \\
\textbf{Attendees:} \\
Cilla, Emma, Jean-Paul, Nattarintra, Linnea, Malte Kerl

\subsection*{Discussion and Notes}

The meeting focused on reviewing the feedback on the UML and architectural diagrams and discussing possible technical directions moving forward.

\subsubsection*{UML Diagram and MassTransit}

The group revisited the feedback on the UML diagram. Regarding MassTransit, it was noted that a message class was missing in the diagram. Since MassTransit is a message-based framework, this made its intended use unclear.

Malte stated that he is not very familiar with MassTransit himself and would need to look further into it if the group decides to use it. The group explained the earlier challenges experienced with Redis and Celery and discussed whether MassTransit could potentially alleviate some of those issues.

It became clear during the discussion that MassTransit would not solve all of the previously encountered problems, and that Redis or Celery would likely still be required. Malte suggested that creating a small prototype could be a good way to evaluate whether MassTransit is a viable path forward.

Malte also asked whether the group feels comfortable using MassTransit given the limited timeframe of the project. The majority of the group expressed that the solution currently feels overwhelming and potentially too complex.

\subsubsection*{Simplifying the Architecture}

To reduce complexity and save time, Malte suggested that persistence could be deprioritized. Instead, an in-memory solution could be used, meaning that the crawl queue does not need to be persisted and can be dropped between executions.

The group continued to discuss Redis, Celery, MassTransit, and RabbitMQ. RabbitMQ was highlighted as a possible alternative, as it is compatible with C\#, which fits well with the project. One suggestion from Malte was to continue using RabbitMQ in an in-memory setup.

\subsubsection*{Team Structure and Prototyping}

Another proposal from Malte was to divide the group into two teams:
\begin{itemize}
    \item One team focusing on query processing.
    \item One team focusing on the crawler and database.
\end{itemize}

It was suggested that the teams rotate roles midway through the project so that all group members get exposure to all parts of the system.

The architecture was acknowledged as a good starting point, but given the limited time available (approximately 20 hours per week for this course), it is important to work efficiently. Several suggestions were given to help save time and reduce the feeling of being overwhelmed.

The group agreed that working in an agile and iterative manner would be beneficial. Starting with prototypes was recommended, potentially one prototype per team. These prototypes can then be iteratively improved and expanded during the project.

\subsubsection*{Proposed Initial Work Distribution}

A possible approach for the initial phase was discussed:
\begin{itemize}
    \item \textbf{Team 1 (Crawler):} Implement a basic crawler prototype that downloads a page, extracts links, and places them into a message.
    \item \textbf{Team 2 (Query Processing):} Implement basic query processing, create a small index containing approximately 20--30 terms, and possibly generate dummy data that is available in the database at application startup.
\end{itemize}

It was also noted that some form of simple UI is needed in order to view and verify query results.

\subsection*{Next Meeting}
The next supervisor meeting is scheduled for \textbf{Thursday, January 22 at 15:00}.

% ==========================================
%           MEETING END
% ==========================================
\newpage
% ==========================================
%           MEETING START
% ==========================================

\section*{Meeting: 2026-01-15 (Sync Meeting)}

\textbf{Time:} 09:00 \\
\textbf{Purpose:} Synchronization meeting prior to the supervisor meeting with Malte

\textbf{Attendees:} \\
Cilla, Emma, Jean-Paul, Nattarintra, Linnea

\subsection*{Discussion and Notes}

The meeting focused on reviewing the feedback received from Malte regarding the updated UML and architectural diagrams. Jean-Paul presented the feedback, which was discussed and clarified within the group.

\subsubsection*{Summary of Feedback from Malte}

\textbf{What looks good:}
\begin{itemize}
    \item The \textit{Backend\_Core} and \textit{Backend\_Infrastructure} modules are overall well-structured and appear well thought out.
\end{itemize}

\textbf{Unclear aspects / questions:}
\begin{itemize}
    \item The distinction between \textit{Backend\_Application} and \textit{Backend\_Infrastructure} feels somewhat arbitrary, particularly regarding the responsibility of the \textit{ICrawlJobRepository}.
    \item The database module is unclear:
    \begin{itemize}
        \item It is not obvious what the module is intended to represent.
        \item It is unclear whether the diagram is complete or merely a placeholder.
    \end{itemize}
\end{itemize}

\textbf{MassTransit-related feedback:}
\begin{itemize}
    \item Malte is not personally familiar with MassTransit, but based on his understanding:
    \begin{itemize}
        \item MassTransit is a message-based framework.
        \item The current UML diagram does not clearly show messages or producers, making the use of MassTransit difficult to understand.
    \end{itemize}
\end{itemize}

\textbf{Concrete technical issues to address:}
\begin{itemize}
    \item A crawler interface is missing, even though the crawler is typed as \textit{ICrawler}.
    \item There is no class implementing \textit{IDomainV}.
    \item The members of \textit{StartCrawlUseCase} appear odd or inconsistent and should be reviewed.
\end{itemize}

\subsubsection*{General Discussion}

The group agreed that the feedback is valuable and that the upcoming supervisor meeting with Malte will be used to ask follow-up questions and clarify how to proceed in order to move the project forward.

Jean-Paul has also started working on defining use cases for the crawler in GitHub, including creating corresponding issues.

Nattarintra raised a question regarding how many retry attempts the crawler should perform. The group agreed that \textbf{three retries} should be sufficient before the URL is removed from the database.

The group discussed whether test cases and the classes they relate to should be included in the same GitHub issue or separated into different issues. It was decided to raise this question with Malte during the supervisor meeting.

It was also agreed that additional research on MassTransit is needed in order to better understand how it could support the project architecture.

\subsection*{Action Items}
\begin{itemize}
    \item Use the supervisor meeting with Malte to clarify architectural questions and next steps.
    \item Continue refining UML and architecture based on received feedback.
    \item Perform additional research on MassTransit and its role in the project.
\end{itemize}

% ==========================================
%           MEETING END
% ==========================================
\newpage
% ==========================================
%           MEETING START
% ==========================================

\section*{Meeting: 2026-01-14 (Daily Stand-up)}

\textbf{Attendees:} \\
Emma, Cilla, Jean-Paul, Nattarintra, Linnea

\subsection*{Discussion and Notes}
The meeting was held as a daily stand-up. Each team member briefly reported what they worked on previously and what they will continue working on.

\begin{itemize}
    \item \textbf{Emma} did not work on any pull requests yesterday since all available PRs had already been assigned. Instead, she researched Jean-Paul’s proposal to use MassTransit instead of Redis and Celery. She also reviewed Jean-Paul’s updated UML diagram and added comments. Today, Emma will review Cilla’s suggested changes in an issue and continue working towards completing the FRQ-1000 series with corresponding test cases, issues, and pull requests.
    
    \item \textbf{Cilla} received feedback on issue FRQ-1009, where it was suggested that two unit tests may be required. The group discussed this issue and agreed to keep FRQ-1009 as it is, but to design two test cases that together capture the intent of the requirement. The issue may be revisited later if needed. Cilla will continue working on this today.
    
    \item \textbf{Jean-Paul} created a new draft of the UML diagram, introducing MassTransit as a solution for queue handling. MassTransit addresses the project’s queuing needs and is compatible with Entity Framework, making it suitable for the project. Today, Jean-Paul will refine details in the diagram and send the draft to Malte for feedback and clarification regarding the use of MassTransit. If time permits, he will also take on a pull request.
    
    \item \textbf{Nattarintra} conducted research on Redis and Celery previously, but will pause this work since the group is now focusing on MassTransit instead. Today, she will review the updated UML diagram and prepare for the upcoming supervisor meeting.
    
    \item \textbf{Linnea} cleaned up and published meeting notes and added comments to issues where test cases may be testing multiple concerns within a single test. She also assigned herself to a pull request and began reviewing it. Today, Linnea will continue reviewing PRs and publish the meeting notes from today’s meeting.
\end{itemize}

The group noted that a decision regarding responsibility for meeting notes had not been finalized previously. It was agreed that meeting notes will be handled as a separate role. Starting next week, a \textit{secretary} will be appointed on a weekly basis, and this role will rotate among group members. For the current week, Linnea will continue writing all meeting notes.

\subsection*{Next Meetings}
A short synchronization meeting will be held tomorrow, January 15 at 09:00, followed by the supervisor meeting with Malte at 10:00.

% ==========================================
%           MEETING END
% ==========================================
\newpage
% ==========================================
%           MEETING START
% ==========================================

\section*{Meeting: 2026-01-13 (Daily Stand-up)}

\textbf{Attendees:} \\
Cilla, Nattarintra, Jean-Paul, Emma, Linnea

\subsection*{Discussion and Notes}
A daily stand-up meeting was held where each team member briefly reported on what they worked on previously and what they are currently working on.

\begin{itemize}
    \item \textbf{Cilla} was absent the previous day due to illness. She will continue working on the test case she started last week.
    \item \textbf{Nattarintra} continued working on test cases FRQ-3004 and FRQ-3006 and has conducted research on Redis and Celery. She will continue this research today.
    \item \textbf{Jean-Paul} has researched Redis and Celery, created several pull requests, and reviewed pull requests from other team members. He would like to further discuss ideas related to Redis and Celery and how the team should proceed.
    \item \textbf{Emma} has continued working on pull requests and has written review comments with suggestions for improvements. She will continue reviewing the remaining pull requests.
    \item \textbf{Linnea} has finalized and documented meeting notes from the previous meeting and created a pull request for these notes. She has also reviewed all currently open pull requests and created a prioritization table, available in OneNote under the tab \textit{``Needing to be researched''}. Linnea will assign herself to one or more pull requests for review and will also research Redis and Celery to gain a better overall understanding.
\end{itemize}

The team discussed whether an additional role, such as a Lead Developer, was needed at this stage. It was agreed that this role is not required at the moment, but the topic will be revisited later if the need arises as development progresses.

\subsection*{Focus and Next Steps}
The main focus for the week remains on the crawler and the 1000-series requirements.

\subsection*{Next Meeting}
The next meeting will be held on January 14th at 08:30 as a daily stand-up.

% ==========================================
%           MEETING END
% ==========================================

\newpage
% ==========================================
%           MEETING START
% ==========================================

\section*{Meeting: 2026-01-12, 13:10}

\textbf{Attendees:} \\
Linnea, Emma, Jean-Paul, Nattarintra

\subsection*{Context}
The meeting focused on synchronizing current work, confirming responsibilities for the coming days, and establishing a shared prioritization model for GitHub issues and pull requests.

\subsection*{Current Work Status}
Each team member briefly reported on their current tasks and plans for the day:
\begin{itemize}
    \item \textbf{Jean-Paul} continues working on the UML and class diagrams.
    \item \textbf{Emma} and \textbf{Nattarintra} continue working on the GitHub tasks they started the previous week.
    \item \textbf{Linnea} will review all pending pull requests and establish a priority order for which PRs must be approved before Thursday.
\end{itemize}

Once a pull request is considered ready for approval, team members may self-assign and proceed with review and merge.

\subsection*{Prioritization Model}
The group agreed on a shared priority model for GitHub issues and tasks:
\begin{itemize}
    \item \textbf{P0:} High priority
    \item \textbf{P1:} Medium priority
    \item \textbf{P2:} Low priority
\end{itemize}

When creating a new issue on GitHub, a priority level should be assigned. Priorities may be adjusted as the work progresses.

\subsection*{Next Meeting}
\begin{itemize}
    \item \textbf{Daily Standup:} Tuesday, January 13 at 13:00
    \item Duration: Maximum 15 minutes
\end{itemize}

% ==========================================
%           MEETING END
% ==========================================

\newpage

% ==========================================
%           MEETING START
% ==========================================

\section*{Meeting: 2026-01-09}

\textbf{Attendees:} \\
Jean-Paul, Emma, Nattarintra, Linnea, Cilla

\subsection*{Context}
This was a short follow-up meeting held after the supervisor meeting, with the purpose of synchronizing roles and tasks ahead of the next supervisor meeting.

\subsection*{Key Deadlines}
\begin{itemize}
    \item \textbf{UML and Class Diagrams:} Must be completed before the next supervisor meeting on \textbf{Thursday, January 15 at 15:00}.
\end{itemize}

\subsection*{Roles and Responsibilities}
The following roles and task distribution were agreed upon:

\begin{itemize}
    \item \textbf{Project Lead: Linnea}
    \begin{itemize}
        \item Review pending Pull Requests.
        \item Create a priority list identifying PRs that block project progress.
        \item Highlight crawler-related PRs as highest priority.
        \item Team members will self-assign PRs for review and approval.
    \end{itemize}

    \item \textbf{UML and Class Diagram:}
    \begin{itemize}
        \item Jean-Paul and Cilla will work on the UML Class Diagram for the crawler.
    \end{itemize}

    \item \textbf{GitHub Tasks:}
    \begin{itemize}
        \item Nattarintra and Emma will continue working on tasks and issues available in the GitHub repository.
    \end{itemize}
\end{itemize}

It was agreed that roles and responsibilities will rotate between group members to ensure shared understanding and balanced workload.

\subsection*{Testing Strategy}
The group discussed test case design and agreed on the following principles:
\begin{itemize}
    \item Test cases should be as focused and clean as possible.
    \item Each test case should verify \textbf{one specific behavior or requirement}.
    \item The goal is to enable clear and maintainable unit testing in the implementation.
\end{itemize}

\subsection*{Technical Discussions}
\begin{itemize}
    \item The group agreed to investigate suitable frameworks or approaches for handling \textbf{normalized words}.
    \item Jean-Paul reviewed the ER diagram to ensure that all group members had a clear understanding of the data model.
\end{itemize}

% ==========================================
%           MEETING END
% ==========================================
\newpage
% ==========================================
%        SUPERVISOR MEETING START
% ==========================================

\section*{Supervisor Meeting: 2026-01-08}

\textbf{Supervisor:} Malte Kerl \\
\textbf{Attendees:} Emma, Cilla, Jean-Paul, Nattarintra, Linnea \\

\subsection*{Feedback on ER Diagram}

\begin{itemize}
  \item A question was raised regarding how the system represents pages that are no longer available.
  One possible solution is to simply remove such pages from the database. The solution does not need to be sophisticated, but the group should be able to reason about and motivate the chosen approach.

  \item The use of a \textbf{normalized representation of words} was recommended. At the moment, several different representations of the same word may exist. The purpose and benefit of normalization should be clearly understood and applied.

  \item Storing a \textbf{datetime for last crawled} was confirmed to be a good design choice.

  \item Clarification was requested regarding the purpose of \textbf{HTTP status}.
  If it is used to indicate removed or redirected pages (e.g.\ HTTP 302), then it is reasonable to keep it.

  \item The attribute \texttt{priority\_integer} can be removed for now, as it is not necessary.
  Similarly, the \texttt{Done} status in the crawl queue can be removed.

  \item Overall, the ER diagram is considered largely complete.

  \item For handling concurrency in the crawler, tools such as \textbf{Redis} or \textbf{Celery} may be used (links were provided by Malte).
  The group is encouraged to attempt implementing concurrency. However, if it turns out to be too complex, it is acceptable to skip it.
\end{itemize}

\subsection*{Module Diagram and Architecture}

\begin{itemize}
  \item The module diagram is generally acceptable.
  \item Authentication should be removed.
  \item The paginator should be moved either to the database layer or to \texttt{Backend.Application}.
  \item The role and necessity of a \texttt{BackgroundService} should be reconsidered and clearly motivated.
\end{itemize}

\subsection*{Process and Testing}

\begin{itemize}
  \item The documentation repository and the current use of GitHub issues were reviewed and considered appropriate.

  \item Malte will check with Ulf whether the project should follow SCRUM, and if a backlog is required.
  It may be useful to define a few initial use cases at the beginning, until development of the search engine is underway.

  \item The project does not need to strictly follow Test Driven Development (TDD).
  However, testing is required, and tests should be written continuously alongside development, not postponed until after implementation.
\end{itemize}

\subsection*{Design and Readiness}

\begin{itemize}
  \item The module diagram should be used as the basis for completing the UML and class diagrams.

  \item Malte asked whether the group feels ready to start parallel coding.
  The group expressed that this is not yet the case, as parts of the overall structure and "big picture" are still unclear.
  This is expected to become clearer once the diagrams are finalized.

  \item The query syntax must be defined and documented.

  \item Contact information pages should be crawled at this stage.
  In the future, a solution for overriding \texttt{robots.txt} for such pages will be needed.
\end{itemize}

\subsection*{Next Steps}

\begin{itemize}
  \item Finalize UML and class diagrams and send them to Malte.
  \item Prepare for the next supervisor meeting on \textbf{Thursday, January 15 at 10:00}.
\end{itemize}

% ==========================================
%        SUPERVISOR MEETING END
% ==========================================

\newpage

% ==========================================
%           MEETING START
% ==========================================

\section*{Meeting: 2026-01-08}

\textbf{Attendees:} Cilla, Emma, Jean-Paul, Nattarintra, Linnea \\

\subsection*{Topics Discussed}

\begin{itemize}
  \item The group discussed questions to be raised with the supervisor regarding:
  \begin{itemize}
    \item Whether the project should follow Test Driven Development (TDD).
    \item If TDD would imply additional workload by writing tests before implementation and then iterating.
    \item The need for clarification on what TDD means in practice for this project.
    \item Whether contact information pages should be included in the crawling scope.
    \item With respect to FRQ-1003, whether the crawler should support concurrent crawling.
    \item If concurrency increases complexity or execution time.
    \item The consequences of not using concurrency.
  \end{itemize}

  \item It was noted that the group should organise the work in a more structured way, including clearer role and task distribution.

  \item The documentation is not yet complete. There are still test cases that need to be written for several requirements.

  \item On GitHub, issues exist for most backend-related tasks, while no issues have yet been created for frontend work.
\end{itemize}

\subsection*{Task Distribution}

\begin{itemize}
  \item Jean-Paul will continue writing test cases for crawler-related requirements (RQ 1000-series).
  \item Cilla has also started working on crawler test cases and will continue contributing in that area.
  \item Nattarintra will begin working on requirements and test cases in the RQ 3000-series.
  \item Emma, Cilla, and Linnea will conduct a short walkthrough of the development environment in VS Code to investigate and resolve issues encountered when running the program.
\end{itemize}

\subsection*{Next Steps}

\begin{itemize}
  \item Model the crawler and incorporate feedback from the supervisor meeting later the same day (15:00).
  \item Work towards clearer role definitions and task distribution within the group.
\end{itemize}

% ==========================================
%           MEETING END
% ==========================================

\newpage

% ==========================================
%           MEETING START
% ==========================================

\section*{Meeting: 2026-01-05} % <--- Change Date Here

\textbf{Attendees:} \\
Emma, Cilla, Jean-Paul, Nattarintra

\subsection*{Agenda}
\begin{enumerate}[noitemsep]
    \item What was done during the holidays
    \item Divide work between members.
\end{enumerate}

\subsection*{Discussion and Notes}
We discussed what we have been doing during holidays and how we should continue with the project.


\subsection*{Action Items}
\begin{itemize}
    \item \textbf{JP:} Summary of diagram 
    \item \textbf{Emma:} Write testcase requirements
    \item \textbf{Cilla:} Write testcase requirements
    \item \textbf{Nattarintra:} Write testcase requirements
\end{itemize}

% ==========================================
%           MEETING END
% ==========================================

\newpage

% ==========================================
%           MEETING START
% ==========================================

\section*{Meeting: 2025-12-19} % <--- Change Date Here

\textbf{Attendees:} \\
Emma, Linnea, Jean-Paul

\subsection*{Agenda}
\begin{enumerate}[noitemsep]
    \item What was done the day before.
    \item Discuss reporting.
    \item Divide work between members.
\end{enumerate}

\subsection*{Discussion and Notes}
Everyone present mentioned what they have done they day before.
We discussed reporting and decided to send our current activity report in conjunction with a small "to do next week" list.


\subsection*{Action Items}
\begin{itemize}
    \item \textbf{JP:} Review meeting notes PR, Tests
    \item \textbf{Emma:} What to include in an index, Review Query syntax PR
    \item \textbf{Linnea:} How to limit crawler rate limit
    \item \textbf{Cilla:} Investigate different algorithms for searching
    \item \textbf{Nattarintra:} Test
\end{itemize}

% ==========================================
%           MEETING END
% ==========================================

\newpage

% ==========================================
%           MEETING START
% ==========================================

\section*{Meeting: 2025-12-18} % <--- Change Date Here

\textbf{Attendees:} \\
Emma, Cilla, Linnea, Jean-Paul, Nattarintra

\subsection*{Agenda}
\begin{enumerate}[noitemsep]
    \item Divide work between members
\end{enumerate}

\subsection*{Discussion and Notes}
We focused on dividing the tasks for the search engine project.
Everyone has been assigned a specific area to research or implement.

\subsection*{Action Items}
\begin{itemize}
    \item \textbf{JP:} Query syntax
    \item \textbf{Emma:} What to include in an index, upload meetingnotes to Git Repository
    \item \textbf{Linnea:} How to limit crawler rate limit
    \item \textbf{Cilla:} Investigate different algorithms for searching
    \item \textbf{Nattarintra:} Test
\end{itemize}

% ==========================================
%           MEETING END
% ==========================================

\newpage

% ==========================================
%           MEETING START
% ==========================================

\section*{Meeting: 2025-12-17}
\textbf{Context:} Updated directives from Malte (replaces parts of original spec).

\textbf{Attendees:} \\
Emma, Cilla, Linnea, Jean-Paul, Nattarintra

\subsection*{1. Technical Requirements (Scope)}
\begin{itemize}
    \item \textbf{Database:} 
    \begin{itemize}
        \item Free choice (Postgres, MariaDB, SQLite are all okay).
        \item \textit{Important:} Avoid MS SQL Server (due to Linux compatibility issues). Choose something easy to install/run.
        \item Storage: 512 GB is available on the VM, but we likely only need a few hundred MB.
    \end{itemize}
    
    \item \textbf{Web Crawler Behavior:}
    \begin{itemize}
        \item \textbf{Whitelist:} Strict. Only crawl \texttt{ltu.se} and specific links provided. Do \textbf{not} follow external links (e.g., Facebook, Instagram).
        \item \textbf{Robots.txt:} Generally respect it, \textit{unless} it blocks "contacts" or "users". We must index contact pages so staff can be found.
        \item \textbf{Content:} Skip images to reduce complexity. Focus on text.
        \item \textbf{Rate Limiting:} No hard limit, but "don't DDoS". A few hundred requests per minute is acceptable.
    \end{itemize}

    \item \textbf{Search Algorithm \& Ranking:}
    \begin{itemize}
        \item Use a combination of \textbf{PageRank} (link popularity) and \textbf{Word Frequency}.
        \item We are encouraged to experiment with algorithms to find the best results.
    \end{itemize}
\end{itemize}

\subsection*{2. Documentation \& Requirements (Milestone 0)}
\begin{itemize}
    \item \textbf{Requirements Document:} 
    \begin{itemize}
        \item Add a \textbf{third column} for "Test Cases". (Be specific: How do we prove the requirement is met?)
        \item Remove "Test Coverage" from the requirements doc (keep it in code, but not as a functional requirement).
    \end{itemize}
    \item \textbf{Diagrams:} Create initial drafts for UML Diagrams (e.g. a Class Diagram), ER-Diagram (Database). Share these early ("Share everything").
    \item \textbf{Tools:} Documentation in LaTeX, Code in Git.
\end{itemize}

\subsection*{3. Project Structure \& Team Roles}
Total team size: 5 members.
\begin{itemize}
    \item \textbf{1 Lead Developer} (Responsible for overall architecture).
    \item \textbf{2 Frontend Developers}.
    \item \textbf{2 Backend Developers}.
\end{itemize}
\textbf{Workflow:} Use Code Reviews and Pull Requests (PR). Clear separation of tasks between Frontend and Backend.

\subsection*{4. Reporting \& Next Steps}
\begin{itemize}
    \item \textbf{Weekly Report:} Send an email \textbf{every Friday}.
    \begin{itemize}
        \item Content: "Activity report" (what we did) and "Next to-dos" (what we will do).
        \item \textit{Note:} Send one immediately before the holidays.
    \end{itemize}
    \item \textbf{Next Meeting:} January 8th at 15:00.
\end{itemize}

% ==========================================
%           MEETING END
% ==========================================

\newpage
% ==========================================
%           MEETING START
% ==========================================

\section*{Meeting: 2025-12-16} % <--- Change Date Here

\textbf{Attendees:} \\
Emma, Cilla, Linnea, Jean-Paul, Nattarintra

\subsection*{Agenda}
\begin{enumerate}[noitemsep]
    \item Meeting for preparation 
\end{enumerate}

\subsection*{Discussion and Notes}
We had a shorter meeting to prepare questions regarding tomorrows meeting with Malte.

% ==========================================
%           MEETING END
% ==========================================

\newpage

% ==========================================
%           MEETING START
% ==========================================
\section*{Meeting: 2025-12-15} % <--- Change Date Here

\textbf{Attendees:} \\
Emma, Cilla, Linnea, Jean-Paul, Nattarintra

\subsection*{Agenda}
\begin{enumerate}[noitemsep]
    \item Meeting for preparation 
\end{enumerate}

\subsection*{Discussion and Notes}
We had a shorter meeting to prepare questions regarding tomorrows meeting with Malte.
% ==========================================
%           MEETING END
% ==========================================

\newpage
% ==========================================
%           MEETING START
% ==========================================

\section*{Meeting: 2025-12-11} % <--- Change Date Here

\textbf{Attendees:} \\
Emma, Cilla, Linnea, Jean-Paul, Nattarintra

\subsection*{Agenda}
\begin{enumerate}[noitemsep]
    \item Daily meeting we discussed what to focus on, Jean-Paul set up a Req document and we started to fill it up whit Test and priorities them. 
\end{enumerate}

\subsection*{Discussion and Notes}
We had a shorter meeting to prepare questions regarding tomorrows meeting with Malte.
% ==========================================
%           MEETING END
% ==========================================

\newpage


% ==========================================
%           MEETING START
% ==========================================

\section*{Meeting: 2025-12-10 (Project Kick-off)}
\textbf{Attendees:} Team and Malte (Client)

\subsection*{Agenda}
\begin{enumerate}[noitemsep]
    \item Review of project scope and limitations
    \item Technical requirements clarification
    \item Expectations for Milestone 0
\end{enumerate}

\subsection*{Decisions \& Directives (Summary)}

\subsubsection*{1. Scope \& Crawler Behavior}
\begin{itemize}
    \item \textbf{Whitelist:} Strictly \texttt{*.ltu.se}. Do not follow external links (Facebook, etc.). Future domains like \texttt{islab.se} might be added later.
    \item \textbf{Robots.txt:} strictly adhere to it for Milestone 0. (Exceptions for contact pages might be added later).
    \item \textbf{Dynamic Content:} LTU's dynamic content is server-side rendered, so no "headless browser" is required.
    \item \textbf{PDFs:} Crawler must detect PDF links. Indexing the \textit{content} of PDFs is optional (nice-to-have).
\end{itemize}

\subsubsection*{2. Search Functionality}
\begin{itemize}
    \item \textbf{Queries:} Optimized for simple queries (2--3 words). Ranking priority: All terms $>$ Some terms $>$ One term.
    \item \textbf{Pagination:} Required (e.g., 10, 20 results per page).
    \item \textbf{Performance:} Response time should be under 10 seconds.
    \item \textbf{Optional features:} Wildcards (*), category filters, and boolean operators are \textit{not} mandatory.
\end{itemize}

\subsubsection*{3. Technology \& Deployment}
\begin{itemize}
    \item \textbf{UI:} Plain HTML/CSS/JS is perfectly acceptable. No complex frameworks (React/Vue) needed unless desired.
    \item \textbf{Deployment:} Local execution is enough. No server deployment required yet.
\end{itemize}

\subsection*{Deliverables for Milestone 0}
\begin{itemize}
    \item \textbf{Requirements Document:} Must include Stakeholder requirements, Functional requirements, and Non-functional requirements.
    \item \textbf{Diagrams:} UML Class diagram + ER Diagram (Database).
\end{itemize}

% ==========================================
%           MEETING END
% ==========================================

\newpage

% ==========================================
%           MEETING START
% ==========================================

\section*{Meeting: 2025-12-09} % <--- Change Date Here

\textbf{Attendees:} \\
Emma, Cilla, Linnea, Jean-Paul, Nattarintra

\subsection*{Agenda}
\begin{enumerate}[noitemsep]
    \item Get to know eachother
    \item Send a mail to Malte
    \item Set up a Github project
\end{enumerate}

\subsection*{Discussion and Notes}
We had a presentation round and talked about the project and sent a email to Malte for scheduling a meeting. Finally we set up a Github Project

% ==========================================
%           MEETING END
% ==========================================
\end{document}